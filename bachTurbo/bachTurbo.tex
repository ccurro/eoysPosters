\documentclass{article}
\usepackage{xcolor}
\usepackage{microtype}
\usepackage[paperwidth=11in,paperheight=8in,top=0.5in,bottom=0.5in,left=0.5in,right=0.82in]{geometry}
\usepackage{fontspec}
\pagestyle{empty}
\setlength{\parindent}{0pt}

% DIN 1451 Std or DIN 30640 Std for descriptions
% Trade Gothic LT for titles
\begin{document}
{

	\fontspec{Trade Gothic LT}[Color=red]
	\fontsize{0.65in}{0.6in}\selectfont 
	\bfseries
	TURBOJET ENGINE TEST STAND
}

\vspace{0.25in}

{
	\fontspec{DIN 1451 Std}
	\fontsize{0.55in}{0.6in}\selectfont
	DANIEL BACH {\fontsize{0.4in}{0.6in}\selectfont AND} GARY GO {\fontsize{0.4in}{0.6in}\selectfont ME '15}
}

\vspace{0.1in}

{
	\fontspec{DIN 1451 Std}
	\fontsize{0.4in}{0.4in}\selectfont
	ADVISOR: GEORGE DELAGRAMMATIKAS
}

\vspace{0.2in}

{
	\fontspec{DIN 1451 Std}[Ligatures=TeX]
	\fontsize{0.41in}{0.46in}\selectfont

%	We designed, specced, and constructed a test stand for a turbojet engine.

	Turbojet engines are widely used as teaching tools in engineering
	education.  The Cooper Union Automotive Lab possesses a small turbojet
	engine for experimentation purposes.  A test stand is necessary to conduct
	experiments with this engine.  Commercial test stands are very costly.  A
	low-cost custom test stand was designed and built to provide a platform
	for turbojet experimentation.  First, a baseline test stand was
	constructed to get the engine up and running.  After a few successful test
	runs, the test stand was modified to incorporate a load cell for measuring
	thrust.  Preliminary data was collected, proving the functionality of the
	system.  Future work will bring even greater capability to the test stand,
	making it more suitable for research and testing that will supplement the
	Cooper Union mechanical engineering curriculum.

}

\end{document}