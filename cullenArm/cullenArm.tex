\documentclass{article}
\usepackage{xcolor}
\usepackage{microtype}
\usepackage[paperwidth=11in,paperheight=8in,top=0.5in,bottom=0.5in,left=0.5in,right=0.82in]{geometry}
\usepackage{fontspec}
\pagestyle{empty}
\setlength{\parindent}{0pt}

% DIN 1451 Std or DIN 30640 Std for descriptions
% Trade Gothic LT for titles
\begin{document}
{
	\fontspec{Trade Gothic LT}[Color=red]
	\fontsize{0.8in}{0.4in}\selectfont 
	\bfseries
	ARM
}

\vspace{0.25in}

{
	\fontspec{DIN 1451 Std}
	\fontsize{0.494in}{0.6in}\selectfont
	HARRISON CULLEN, ALYSSA DAVIS {\fontsize{0.4in}{0.6in}\selectfont AND}
	 JENNIFER TASHMAN {\fontsize{0.4in}{0.6in}\selectfont BSE '15}
}

\vspace{0.1in}

{
	\fontspec{DIN 1451 Std}
	\fontsize{0.4in}{0.4in}\selectfont
	ADVISOR: TOBY CUMBERBATCH
}

\vspace{0.2in}

{
	\fontspec{DIN 1451 Std}[Ligatures=TeX]
	\fontsize{0.4in}{0.46in}\selectfont

    Arm is a life-sized endoskeletal structure that mimics the movement of a
	human arm. Computer-controlled linear and rotary actuators are used to operate
	Arm, and the automaton is programmed to perform and communicate its human-like
	features. Freud’s theory of the Uncanny was employed in order to contextualize
	aspects of a biomimetic robot based on the response evoked from a human
	audience. Arm’s main endoskeletal structure is made entirely of brass, and is
	manufactured in a manner that seeks to manipulate simple motor functions to
	subsequently produce a more nuanced and believable representation of human
	motion. All moving parts are exposed in order to provide an easily
	understandable, and ultimately more inviting, mechanically performative
	experience.

}

\end{document}