\documentclass{article}
\usepackage{xcolor}
\usepackage{microtype}
\usepackage[paperwidth=11in,paperheight=7in,top=0.5in,bottom=0.5in,left=0.5in,right=0.82in]{geometry}
\usepackage{fontspec}
\pagestyle{empty}
\setlength{\parindent}{0pt}

% DIN 1451 Std or DIN 30640 Std for descriptions
% Trade Gothic LT for titles
\begin{document}
{
	\fontspec{Trade Gothic LT}[Color=red]
	\fontsize{0.8in}{0.4in}\selectfont 
	\bfseries
	VACUUM TUBE PRE-AMP
}

\vspace{0.25in}

{
	\fontspec{DIN 1451 Std}
	\fontsize{0.55in}{0.6in}\selectfont
	AVI GADISH {\fontsize{0.4in}{0.6in}\selectfont EE '15 AND} DENNIS GAVRILOV {\fontsize{0.4in}{0.6in}\selectfont EE '16}
}

\vspace{0.1in}

{
	\fontspec{DIN 1451 Std}
	\fontsize{0.4in}{0.4in}\selectfont
	ADVISOR: TIM HOERNING
}

\vspace{0.2in}

{
	\fontspec{DIN 1451 Std}[Ligatures=TeX]
	\fontsize{0.397in}{0.46in}\selectfont

		In satisfaction of a ECE314 ``Audio Engineering Projects'' project
		utilizing antiquated technologies, a vacuum tube pre-amplifier circuit
		was designed and prototyped. A vacuum tube is a voltage amplifier,
		which was slowly phased out after the invention of the transistor.
		Vacuum tubes distort waveforms differently, and more harmonically,
		than a transistor, which is the reason vacuum tubes are still sought
		after by audiophiles around the world. In addition to `Gain' and
		`Level' control, the design of the Tone Stack from the Fender Super
		Reverb guitar amplifier was modified and implemented in our pre-amp.

}

\end{document}